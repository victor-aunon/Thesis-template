%*****************************************************************************%
%*****************************************************************************%
%********************* CONFIGURATION FILE - PREAMBLE *************************%
%*****************************************************************************%
%*****************************************************************************%

\listfiles
% places a list of the files used and their version numbers in the log file

%*****************************************************************************%
%******************************** CONSTANTS **********************************%
%*****************************************************************************%

\newif\ifcustomFonts
% Set customFonts to true if you want to use any OTF font in your system
\customFontstrue % or \customFontsfalse for false

% Compiler to be used - XETEX IF YOU ARE USING FONTSPEC
\ifcustomFonts
    \newcommand{\compiler}{xetex} % xetex, pdftex, etc.
\else
    \newcommand{\compiler}{pdftex} % pdftex, pdftex, etc.
\fi

% Thesis title
\newcommand{\thesisTitle}{The fabulous name of my thesis}
% Institution logo location
\newcommand{\logoLocation}{logos/logo.pdf}
% Department name
\newcommand{\department}{My department or research center}
% Author
\newcommand{\authorMe}{Mr. Ángel Auñón}
% City and date
\newcommand{\cityDate}{My city, October 2020}

% Supervisors
\newcommand{\supervisorOne}{Dr. Supervisor 1}
\newcommand{\supervisorTwo}{Dr. Supervisor 2} % may be empty
\newcommand{\supervisorThree}{Dr. Supervisor 3} % may be empty

% Examiners
\newcommand{\examinerOne}{Dr. examiner 1}
\newcommand{\examinerTwo}{Dr. examiner 2}
\newcommand{\examinerThree}{Dr. examiner 3}

% Defense committee
\newcommand{\chairman}{Dr. Wonderful Chairman}
\newcommand{\secretary}{Dr. Awesome Secretary}
\newcommand{\member}{Dr. Incredible Member}

% Paragraph font size
\newcommand{\fontSize}{11pt}
% Paragraph line spacing
\newcommand{\lineSpacing}{16.5pt} % \fontSize * 1.5 (1.1, 1.2, 1.3, ...)

% Bibliography file location
\newcommand{\bibliographyFile}{bibliography/bibliography.bib}

% Hyperref options
\newcommand{\linkColor}{blue}
\newcommand{\linkColorTOC}{black} % Allows you to use another color for the 
                                  % TOC links, see minitocs in chapter 1 & 2
\newcommand{\citeColor}{blue}
\newcommand{\fileColor}{magenta}
\newcommand{\urlColor}{blue}

% Fonts
% If you want to load LaTeX typeface fonts take a look at the link below to
% see their names https://www.overleaf.com/learn/latex/font_typefaces e.g. 
% ppl for palatino, ptm for Times, lmss for Latin Modern Sans Serif ...
\newcommand{\mainFont}{Merriweather} % Serif font (preferable) 
\newcommand{\altFont}{Merriweather Sans} % Sans serif font
\newcommand{\numbersFont}{Source Serif Pro}
\newcommand{\mathFont}{TeX Gyre DejaVu Math}
% Useful help for selecting math fonts: https://tex.stackexchange.com/questions/425098/which-opentype-math-fonts-are-available
\newcommand{\chapterFont}{Merriweather Sans}
\newcommand{\sectionFont}{Merriweather Sans}
\newcommand{\subSectionFont}{Merriweather Sans}
\newcommand{\subSubSectionFont}{Merriweather Sans}

% Accent colour
\newcommand{\accentColor}{ForestGreen}
% https://ctan.javinator9889.com/macros/latex/contrib/xcolor/xcolor.pdf

%*****************************************************************************%
%******************************** PACKAGES ***********************************%
%*****************************************************************************%

\usepackage[T1]{fontenc}    % selects T1 as the document font encoding
\usepackage[catalan,spanish,english]{babel} % languages package
\usepackage[utf8]{inputenc} % use UTF-8 input encoding (allows accent marks, ñ)
\usepackage{blindtext}      % package to generate dummy text (demo template)
\usepackage{amsmath}        % Maths package
\usepackage{amssymb}        % Maths symbols package
\ifcustomFonts
\usepackage{fontspec}   % package to use custom fonts
\usepackage{unicode-math} % package to replace math fonts, MUST GO AFTER
% AMSMATH AMSSYMB
\fi
\usepackage{geometry}       % package to define the page margins and sizes
\usepackage{anyfontsize}    % package to modify font size
\usepackage{titlesec}       % package to change titles and subtitles properties
\usepackage[svgnames]{xcolor} % package to use already defined colours
\usepackage{lmodern}        % high quality rendering text
\usepackage{graphicx}       % graphs package
\usepackage{float}          % package to place the figures
\usepackage{fancyhdr}       % package to setup headers and footers
\usepackage{emptypage}      % removes header, footer and page n. in empty pages
\usepackage{lettrine}       % package to create an stylish first letter
\usepackage{caption}        % manages captions properties
\usepackage[english]{minitoc} % package to create mini-TOCs at the beginning of
                            % any chapter
\usepackage{siunitx}        % manages physical units
\usepackage{longtable}      % package to create big tables larger than a page
\usepackage{multirow}       % package to create multirows in tables
\usepackage{multicol}       % package to create multicolumns in tables
\usepackage{tabulary}       % justify text in a table and set auto width
\usepackage{subcaption}     % create subfigures, INCOMPATIBLE WITH SUBFIGURE
\usepackage{tocloft}        % package to modify TOC and LOF appearance
\usepackage[titletoc,toc,title]{appendix} % package to create appendix pages
\usepackage{epigraph}       % package to manage quotes
\usepackage[switch, modulo]{lineno} % package to enumerate the lines, useful
                            % for reviewing. Options-> switch: numbers on the
                            % outer margin, switch*: on the inner margin,
                            % left: on the left margin, right: on the right
                            % margin, modulo: only prints numbers which are
                            % multiple of five,
\usepackage{framed}         % package to create a frame around figures, tables
\usepackage{physics}        % extended functions for writing equations
\usepackage[
    style           = numeric,
    defernumbers    = false,
    backend         = bibtex,
    bibstyle        = numeric,
    firstinits      = true,
    maxnames        = 10,
    block           = space,
    backref         = true,
    sorting         = none
]{biblatex}                 % package to manage the bibliography
\usepackage{xpatch}         % used later to modify the bibliography
\usepackage{xurl}           % package to split long URLs, import after biblatex
\usepackage[
    \compiler,
    bookmarks    = true,
    pdftoolbar   = true,
    pdfmenubar   = true,
    pdffitwindow = false,
    colorlinks   = true,
    linkcolor    = \linkColor,
    citecolor    = \citeColor,
    filecolor    = \fileColor,
    urlcolor     = \urlColor
]{hyperref}                 % package to create links in references, cites and
                            % table of contents (TOC)

%*****************************************************************************%
%************************** PACKAGE ADJUSTMENTS ******************************%
%*****************************************************************************%

%***************************** PAGE MARGINS **********************************%

% Add some extra margin to the gutter
\geometry{bindingoffset=0.75cm}

%******************************* TEXT SPACING ********************************%

% Space between paragraphs
\parskip=0.4cm
% Extra space between rows in tables
\renewcommand{\arraystretch}{1.25}
% Indentation at the beginning of the paragraph
%\parindent=0mm % commented to keep the indentation
% Space before titles in the table of contents
\setlength{\cftbeforetoctitleskip}{-0.4cm}
% Space before titles in the list of figures
\setlength{\cftbeforeloftitleskip}{-0.4cm}
% Space before titles in the list of tables
\setlength{\cftbeforelottitleskip}{-0.4cm}
% Space before chapters in TOC
\setlength{\cftbeforechapskip}{0.5cm}
% Space before sections in TOC
\setlength{\cftbeforesecskip}{0.2cm}
% Space before subsections in TOC
% Space before subsubsections in TOC
\setlength{\cftbeforesubsecskip}{0.2cm}
\setlength{\cftbeforesubsubsecskip}{0.3cm}
% Space before figures in LOF
\setlength{\cftbeforefigskip}{0.3cm}
% Space before tables in LOT
\setlength{\cftbeforetabskip}{0.3cm}

%******************************* TOC OPTIONS *********************************%

% Maximum mini-TOC depth level
\setcounter{minitocdepth}{4}
% Maximum document depth level
\setcounter{secnumdepth}{4}
% Maximum TOC depth level
\setcounter{tocdepth}{1}

%*************************** BIBLIOGRAPHY OPTIONS ****************************%

% Select bibliography file. Replaces \bibliography command
\addbibresource{\bibliographyFile}

% Remove "in:" - OPTIONAL
\renewbibmacro{in:}{%
\ifentrytype{article}{}{\printtext{\bibstring{in}\intitlepunct}}} 
% http://tex.stackexchange.com/questions/10682/suppress-in-biblatex

% Replace bibliography section name of each chapter - OPTIONAL
\defbibheading{subbibliography}{
    \section*{Chapter~\ref{refsegment:\therefsection\therefsegment} References}}

%*****************************************************************************%
%************************** DEFINE NEW COMMANDS ******************************%
%*****************************************************************************%

% Call the lettrine command with an initial letter 1 line taller
\newcommand{\letr}{\lettrine[lines=1]}
% https://modules.contextgarden.net/dl/t-lettrine/doc/context/third/lettrine/lettrine-doc.pdf

%****************************** FONT OPTIONS *********************************%

% Colour the number of the sections and subsections and rest of subsections
\renewcommand\thesection{\color{\accentColor}\thechapter.\arabic{section}}

\ifcustomFonts
    \setmainfont{\mainFont}
    \setmathfont{\mathFont}
    \newfontfamily{\altfont}{\altFont}
    \newfontfamily{\numbersfont}{\numbersFont}
    \newfontfamily{\chapterfont}{\chapterFont}
    \newfontfamily{\sectionfont}{\sectionFont}
    \newfontfamily{\subsectionfont}{\subSectionFont}
    \newfontfamily{\subsubsectionfont}{\subSubSectionFont}
\else
    \renewcommand\rmdefault{\mainFont} % Sets main serif font
    \newcommand{\altfont}{\fontfamily{\altFont}\selectfont}
    \newcommand{\numbersfont}{\fontfamily{\numbersFont}\selectfont}
    \newcommand{\chapterfont}{\fontfamily{\chapterFont}\selectfont}
    \newcommand{\sectionfont}{\fontfamily{\sectionFont}\selectfont}
    \newcommand{\subsectionfont}{\fontfamily{\subSectionFont}\selectfont}
    \newcommand{\subsubsectionfont}{\fontfamily{\subSubSectionFont}\selectfont}
\fi

% Titlesec usage
% \titleformat{command}[shape]{format}{label}{sep}{before-code}{after-code}
% \titlespacing{command}{left}{before-sep}{after-sep}

% Change chapter color and font
\titleformat{\chapter}[display]{\Huge\bfseries}{\textcolor{\accentColor}%
{\chaptername \, \Huge\thechapter}}{24pt}{\chapterfont\bfseries\fontsize{%
30}{36}\selectfont}
% Chapter top and bottom padding
\titlespacing{\chapter}{0pt}{50pt}{40pt}

% Change section and subsection color and format
% CHANGED IN THEIR DECLARATION, RIGHT IN THE LABEL

% Define a format for TOC, LOT, LOF, Publications, etc. titles & subtitles
\newcommand{\titlestyle}{\bfseries\altfont\fontsize{26pt}{42pt}%
\selectfont}
\newcommand{\subtitlestyle}{\large\bfseries\altfont}

% TOC, LOT, and LOF title fonts
\renewcommand{\cfttoctitlefont}{\titlestyle}
\renewcommand{\cftloftitlefont}{\titlestyle}
\renewcommand{\cftlottitlefont}{\titlestyle}

% TOC fonts
% Chapter number and title
\renewcommand{\cftchappresnum}{\Large\bfseries\textcolor{\accentColor}}
\renewcommand{\cftchapfont}{\bfseries\fontsize{14pt}{21pt}\selectfont}
% Section number and title
\renewcommand{\cftsecpresnum}{\large\bfseries}
\renewcommand{\cftsecfont}{\sectionfont\fontsize{12pt}{18pt}\selectfont}
% Subsection number and title
\renewcommand{\cftsubsecpresnum}{\large\bfseries}
\renewcommand{\cftsubsecfont}{\subsectionfont\fontsize{12pt}{18pt}%
\selectfont}
% Subsubsection number and title
\renewcommand{\cftsubsubsecpresnum}{\large}
\renewcommand{\cftsubsubsecfont}{\subsubsectionfont\fontsize{12pt}{18pt}%
\selectfont}

% Change section, subsection and subsubsection appearances w/o messing the TOC
\titleformat*{\section}{\sectionfont\bfseries\fontsize{16pt}{24pt}\selectfont}
\titleformat*{\subsection}{\subsectionfont\bfseries\fontsize{14pt}{28pt}%
\selectfont}
\titleformat*{\subsubsection}{\subsubsectionfont\fontsize{14pt}{28pt}%
\selectfont}
% Set minitoc sections appearances back to normalfont
\mtcsetfont{minitoc}{section}{\normalfont\small}

%***************************** CAPTIONS OPTIONS ******************************%

\DeclareCaptionFormat{captionstyle}{\itshape{\altfont\small#1#2#3}}
\captionsetup{format=captionstyle, labelfont={color=\accentColor, bf,%
normalsize}}

%***************************** EPIGRAPH OPTIONS ******************************%

% Maximum line width of the quote 
\renewcommand{\epigraphwidth}{0.8\linewidth}
% Quote on the right side of the page
\renewcommand{\epigraphflush}{flushright}
% Text right alignment
\renewcommand{\textflush}{flushright}
% Remove dividing line between quote and author
\renewcommand{\epigraphrule}{0pt}

%******************************* NEW SI UNITS ********************************%

% Usage
% \DeclareSIUnit\myunit{myunit}
